\begin{document}

\item Tiga gaya masing-masing $F_1$ = 20 N, $F_2$ = 15 N, $F_3$ = 12 N bekerja pada batang yang panjangnya L = 40 cm (berat batang diabaikan) seperti pada gambar di bawah
\pilgan{
\item 0,8 N.m
\item 3,2 N.m 
\item 4,0 N.m
\item 7,4 N.m
\item 8,0 N.m
} 


\item Berikut ini pernyataan tentang faktor-faktor gerak rotasi:
\pernyataan{
\item kecepatan sudut
\item letak sumbu rotasi
\item bentuk benda
\item massa benda
}
Faktor yang mempengaruhi besarnya momen inersia adalah . . . .
\pilgan{
\item 1,2,3, dan 4
\item 1,2, dan 3
\item 1,3, dan 4
\item 2,3, dan 4
\item 2 dan 4
}

\item Dua buah bola yang dianggap sebagai partikel dihubungkan dengan seutas tali kawat seperti gambar.

Bila massa bola P dan Q masing-masing 600 g dan 400 g, maka momen ienrsia sistem kedua bola terhadap poros AB adalah . . .
\pilgan{
\item 0,008 kg.m$^2$
\item 0,076 kg.m$^2$
\item 0,124 kg.m$^2$
\item 0,170 kg.m$^2$
\item 0,760 kg.m$^2$
}


\item Sebuah katrol dari benda pejal dengan tali yang dililitkan pada sisi luarnya ditampilkan seperti gambar. Gesekan katrol diabaikan. Jika momen inersia katrol $I=\beta$ dan tali ditarik dengan gaya tetap $F$, maka nila $F$ setara dengan . . . .
\pilgan{
\item $F=\alpha.\beta.R$
\item $F=\alpha.\beta^2.R$
\item $F=\alpha.(\beta.R)^{-1}$
\item $F=\alpha.\beta.R^{-1}$
\item $F=(\alpha.\beta)^{-1}.R$
}
\item Katrol yang bermassa 10 kg dan jari-jarinya 25 cm digantungi massa benda 5 k seperti pada gambar. Mula-mula massa benda diam kemudian dilepaskan maka perceptan sistem katrol adalah . . . ($I_{\text{katrol}} = \frac{1}{2} m.r^2)$
\pilgan{
\item 1,0 m/s$^2$
\item 2,5 m/s$^2$
\item 3,3 m/s$^2$
\item 5,0 m/s$^2$
\item 12,5 m/s$^2$
}

\item Sebuah silinder penjal ($I=\frac{1}{2} m.r^2$) bergerak menggelinding tanpa tergelincir mendaki bidang miring kasar dengan kecepatan awal 10 m/s. Bidang miring itu mempunyai sudut elevasi $\alpha$ dengan sin $\alpha$ = 0,60. Jika percepatan gravitasi $g$ = 10 m/s$^2$ dan kecepatan benda berkurang menjadi 5 m/s, maka jarak yang ditempuh benda itu adalah . . . 
\pilgan{
\item 7,0 m
\item 9,4 m
\item 12,0 m
\item 14,5 m
\item 17,0 m
}

\item Sebuah bendategar berada dalam kesetimbangan rotasi maka . . . .
\pilgan{
\item $\Sigma F_x=0$
\item $\Sigma F_y=0$
\item $\Sigma F_z=0$
\item $\Sigma \tau =0$
\item $\Sigma F=0$ dan $\Sigma \tau=0$
}

\item Urutkan dari gambar-gambar di bawah ini yang termasuk kesetimbangan labil,stabil, atau netral!
\pilgan{ 
\item a,b,c
\item a,c,b
\item b,c,a
\item b,a,c
\item c,a,b
}

\item Perhatikan gambar berikut!

Diketahui panjang batang AB 2,5 m dan berat 200 N serta batang bersandar pada dinding yang licin. Bila sistem batang dalam keadaan setimbang, maka koefisien gesekan batang dengan lantai adalah . . . 
\pilgan{
\item $\frac{1}{3}$
\item $\frac{1}{2}$
\item $\frac{1}{2}\sqrt{2}$
\item $\frac{1}{2}\sqrt{3}$
\item $\frac{1}{3}\sqrt{3}$
}

\item Perhatika gambar bidang berikut!

Koordinat titik berat dari benda tersebut adalah . . . .
\pilgan{
\item (10/4 ; 16/4)
\item (12/4 ; 12/4)
\item (14/4 ; 14/4)
\item (16/4 ; 12/4)
\item (16/4 ; 10/4)
}

\item Seutas kawat yang luas penampangnya 4 mm$^2$ ditarik oleh gaya 3,2 N, sehingga panjangnya bertambah 0,04 mm. Tegangan kawat tersebut adalah . . . .
\pilgan{
\item $ 8\times 10^6$ N/m$^2$
\item $ 7\times 10^6$ N/m$^2$
\item $ 8\times 10^5$ N/m$^2$
\item $ 8\times 10^4$ N/m$^2$
\item $ 5\times 10^4$ N/m$^2$
}

\item Seutas kawat baja yang panjangnya 1,0 m dengan luas penampang 2,0 mm$^2$ digunakan untuk mendukung beban 100 kg. Jika pertambahan panjang kawat baja 2,5 mm, maka regangan kawat tersebut adalah . . . .
\pilgan{
\item $2,5 \times 10^{-3}$
\item $2,5 \times 10^{-2}$
\item $2,5 \times 10^{-1}$
\item $4 \times 10^{2}$
\item $4 \times 10^{3}$
}

\item Sebuah specimen baja berukuran 10 cm x 2 cm x 2 cm ditarik dengan gaya 5000 N bertambah panjang 5 mm. Modulus elastisitas Young bahan tersebut adalah . . . .
\pilgan{
\item $2,5 \times 10^9$ N/m$^2$ 
\item $2,5 \times 10^8$ N/m$^2$
\item $2,5 \times 10^6$ N/m$^2$
\item $4 \times 10^9$ N/m$^2$
\item $4 \times 10^8$ N/m$^2$
}

\item Kawat P dan Q terbuat dari bahan yangsama. Perbandingan antara diameter P dan Q adalah 2 : 3, sedangkan perbandingan antara panjang kawat P dan Q adalah 3 :4 . Dari data tersebut perbadingan antara konstanta gaya kawat P dan Q adalah . .  .
\pilgan{
\item 6 : 12
\item 8 : 9
\item 12 : 6
\item 16 : 27
\item 27 : 16
}

\item Sebuah pegas panjangnya 50 cm dengan konstanta pegas 200 N/m, dipotong menjadi dua bagian yang sama panjang. Potongan pegas tersebut ditarik dengan gaya 40 N akan bertambah panjang sebesar . . . 
\pilgan{
\item 5 cm
\item 10 cm
\item 15 cm
\item 20 cm
\item 25 cm
}

\item Grafik hubungan gaya (F) terhadap perubahan panjang dari percobaan elastisitas pegas di bawah ini.
\begin{tikzpicture}
\draw[stealth-stealth](0,3)node [above]{$F$ (N)}--(0,0)--(3,0) node [right]{$\Delta x$(cm)};
\draw [dashed](0,2) node [left]{16}--(2,2)--(2,0)node [below]{4};
\end{tikzpicture}
Besarnya konstanta elastisitas pegas tersebut adalah . . . 
\pilgan{
\item 0,04 N/m
\item 0,4 N/m
\item 4 N/m
\item 40 N/m
\item 400 N/m
}

\item Sebuah benda bermassa 5 kg  menggantung pada sebuah pegas yang memiliki konstanta pegas sebesar 2.000 N/m. Bila $g$=10 m/s$^2$, pegas tersebut akan bertambah panjang sebesar . . . .
\pilgan{
\item 2,0 cm
\item 2,5 cm
\item 4,0  cm
\item 5,0 cm
\item 6,5 cm
}

\item Dua pegas identik dirangkai paralel dengan konstanta gaya pegas 100 N/m. Jika pada ujung susunan pegas diberi beban 10 N, maka pertambahan panjang masing-masing pegas adalah . . . .

\begin{tikzpicture}
\end{tikzpicture}

\pilgan{
\item 1 m
\item 2 m
\item 3 m
\item 4 m
\item 5 m
}

\item Tiga pegas tersusun seperti gambar berikut. 

Jika tetapan pegas $k_1=k$ dan $k_2=4k$, maka nilai konstanta ($k'$) susunan pegas adalah . . . .
\pilgan{
\item $\frac{3}{4k}$
\item $\frac{3k}{4}$
\item $ \frac{4k}{3}$
\item $3k$
\item $4k$
}

\item Tiga buah pegas A,B, dan C yang identik dirangkai seperti gambar di bawah ini!.

Jika ujung bebas pegas C digantungkan bebab 1,2 N maka sistem mengalami pertambahan panjang 0,6 cm, konstanta masing-masing pegas adalah . . . .
\pilgan{
\item 200 N/m
\item 240 N/m
\item 300 N/m
\item 360 N/m
\item 400 N/m
}
}
\end{document}
